Let $S \subseteq \mathbb{R}^2$ be the set defined by $S = \{(x, y) \in \mathbb{R}_+^2 \mid y \leq \sqrt{x}\}$. Prove that $S$ is a convex set. \defpoints{20}

\textcolor{blue}{Solution}

Let $f(x)=-\sqrt{x}$, then we have
\begin{align*}
f'(x) &= -\dfrac{1}{2\sqrt{x}} \\
f''(x) &= \textcolor{red}{\dfrac{1}{4}}x^{-\frac{3}{2}}
\end{align*}
And since $x\in \mathbb{R}_+$, so we have $f''(x) \geq 0$, which means that $f(x)$ is a convex function.

So $g(x)=\sqrt{x}$ is a concave function.

Then $\forall (x_1, y_1), (x_2, y_2) \in S, \theta\in[0,1]$, we have
\begin{align*}
y_1 &\leq \sqrt{x_1} \\
y_2 &\leq \sqrt{x_2}
\end{align*}
So we have
\begin{align*}
\theta y_1 + (1-\theta)y_2 &\leq \theta\sqrt{x_1} + (1-\theta)\sqrt{x_2} \\
&= \theta g(x_1) + (1-\theta)g(x_2) \\
&\leq g(\theta x_1 + (1-\theta)x_2) \text{\quad\quad (concavity of $g(x)$)} \\
&= \sqrt{\theta x_1 + (1-\theta)x_2}
\end{align*}
Which means that $\theta(x_1, y_1) + (1-\theta)(x_2, y_2) = \left(\theta x_1 + (1-\theta)x_2, \theta y_1 + (1-\theta)y_2\right) \in S$.

So above all, we have proved that $S$ is a convex set.

\newpage